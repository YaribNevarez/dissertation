\documentclass[8pt]{beamer}
\usepackage{graphicx}
\usepackage{booktabs} % For professional looking tables
\usepackage{threeparttable} % For table notes
\usepackage{multirow} % For cells that span multiple rows
\usepackage{glossaries} % In case you use \gls{} in the table
\usepackage{algorithm}
\usepackage{algpseudocode}
\usepackage{amsmath}
\usepackage[backend=biber,style=authoryear]{biblatex}
\addbibresource{../bibliography/mybib}
\usepackage{comment}

% Define glossary entries if not already defined
\newglossaryentry{fp}{name={FP}, description={Floating Point}}

\newcommand{\backupbegin}{
	\newcounter{framenumberappendix}
	\setcounter{framenumberappendix}{\value{framenumber}}
}
\newcommand{\backupend}{
	\addtocounter{framenumberappendix}{-\value{framenumber}}
	\addtocounter{framenumber}{\value{framenumberappendix}} 
}

\mode<presentation> {
	\usetheme{CambridgeUS}
	\usecolortheme{dolphin}
	
	% Remove the navigation symbols
	\setbeamertemplate{navigation symbols}{}
	
	% Explicitly set the headline to be empty
	\setbeamertemplate{headline}{}
	
	% Customize footline to include section and page number
	\setbeamertemplate{footline}{%
		\leavevmode%
		\hbox{%
			\begin{beamercolorbox}[wd=.5\paperwidth,ht=2.25ex,dp=1ex,left]{section in head/foot}%
				\usebeamerfont{section in head/foot}\hspace*{2ex}\insertsection
			\end{beamercolorbox}%
			\begin{beamercolorbox}[wd=.5\paperwidth,ht=2.25ex,dp=1ex,right]{section in head/foot}%
				\usebeamerfont{section in head/foot}\insertframenumber{} / \inserttotalframenumber\hspace*{2ex}
			\end{beamercolorbox}%
		}%
		\vskip0pt%
	}
}




\title[PhD Defense - Yarib Nevarez]{Low-Power Neural Network Accelerators:
	\\Advancements in Custom Floating-Point Techniques}
\author{Yarib Nevarez}
\institute{Universität Bremen}
\date{May 22, 2024}

\begin{document}
	
	{
	\setbeamertemplate{footline}{} % Remove the footline for this slide only
	\begin{frame}
		\begin{figure}
			\includegraphics[width=\textwidth]{slides/figures/cover_image.pdf}
		\end{figure}
		\titlepage % This command creates the title page
		\vfill % This command will push your images to the bottom of the page
		
    \begin{figure}
	\centering % This centers the figure in the slide
	\begin{minipage}[b]{0.24\linewidth} % Adjust the width to accommodate four logos
		\includegraphics[height=1.5em,keepaspectratio]{../figures/UHB_Logo_4c.pdf} % Existing logo 1
	\end{minipage}
	\hfill % This command inserts a space between the minipages
	\begin{minipage}[b]{0.24\linewidth} % Adjust the width for the second logo
		\includegraphics[height=1.5em,keepaspectratio]{../figures/conahcyt.pdf} % Existing logo 2
	\end{minipage}
	\hfill % Space
\begin{minipage}[b]{0.24\linewidth} % New logo space
	\includegraphics[height=2.5em,keepaspectratio]{../figures/NXP_Logo_RGB_Colour.png} % New NXP logo
\end{minipage}
	\hfill % Space
	\begin{minipage}[b]{0.24\linewidth} % Adjust the width for the third logo
		\includegraphics[height=1.5em,keepaspectratio]{../figures/logo_item_ids.png} % Existing logo 3
	\end{minipage}

\end{figure}
		
		\addtocounter{framenumber}{-1} % this decrements the frame counter so the title page doesn't count
	\end{frame}
}

	
	\section*{Introduction}
%\begin{frame}
%	\frametitle{Visibility Control}
%	\only<1>{This text is only on slide 1.}
%	\uncover<2-3>{This text is uncovered on slides 2 and 3.}
%	\visible<2>{This text becomes visible on slide 2.}
%	\invisible<3>{This text becomes invisible on slide 3.}
%\end{frame}

\begin{frame}{Introduction}
	\only<1>{\includegraphics[width=\textwidth]{slides/Introduction/1.pdf}}
	\only<2>{\includegraphics[width=\textwidth]{slides/Introduction/2.pdf}}
	\only<3>{\includegraphics[width=\textwidth]{slides/Introduction/3.pdf}}
	\only<4>{\includegraphics[width=\textwidth]{slides/Introduction/4.pdf}}
	\only<5>{\includegraphics[width=\textwidth]{slides/Introduction/5.pdf}}
	\only<6>{\includegraphics[width=\textwidth]{slides/Introduction/6.pdf}}
	\only<7>{\includegraphics[width=\textwidth]{slides/Introduction/7.pdf}}
	\only<8>{\includegraphics[width=\textwidth]{slides/Introduction/8.pdf}}
	\only<9>{\includegraphics[width=\textwidth]{slides/Introduction/9.pdf}}
	\only<10>{\includegraphics[width=\textwidth]{slides/Introduction/10.pdf}}
	\only<11>{\includegraphics[width=\textwidth]{slides/Introduction/11.pdf}}
	\only<12>{\includegraphics[width=\textwidth]{slides/Introduction/12.pdf}}
	\only<13>{\includegraphics[width=\textwidth]{slides/Introduction/13.pdf}}
	%\only<14>{\includegraphics[width=\textwidth]{slides/Introduction/14.pdf}}
\end{frame}
	
	\section*{Goal and Objectives}
	\begin{frame}
		\frametitle{Goal and Objectives}
		\begin{itemize}
			\item<1-> \textbf{Goal}:
			To establish a future-proof neural network acceleration approach that supports inference and facilitates on-device training for TinyML applications
			
			\vspace{10mm} 
			
			\item<2-> \textbf{Objectives}:
			\begin{itemize}
				\item<3-> Investigate design techniques
				\item<4-> Ensure quality preservation
				\item<5-> Explore on-device optimization
				\item<6-> Evaluate deployment
			\end{itemize}
		\end{itemize}
	\end{frame}

	\begin{frame}
	\frametitle{Outline}
	\tableofcontents % This command automatically creates a table of contents based on the sections and subsections
\end{frame}
	
	\section{Methodology}
\begin{frame}{Methodology}
	\only<1>{\includegraphics[width=\textwidth]{slides/Methodlogy/1.pdf}}
	\only<2>{\includegraphics[width=\textwidth]{slides/Methodlogy/2.pdf}}
	\only<3>{\includegraphics[width=\textwidth]{slides/Methodlogy/3.pdf}}
	\only<4>{\includegraphics[width=\textwidth]{slides/Methodlogy/4.pdf}}
	\only<5>{\includegraphics[width=\textwidth]{slides/Methodlogy/5.pdf}}
	\only<6>{\includegraphics[width=\textwidth]{slides/Methodlogy/6.pdf}}
	\only<7>{\includegraphics[width=\textwidth]{slides/Methodlogy/7.pdf}}
	\only<8>{\includegraphics[width=\textwidth]{slides/Methodlogy/8.pdf}}
	\only<9>{\includegraphics[width=\textwidth]{slides/Methodlogy/9.pdf}}
	\only<10>{\includegraphics[width=\textwidth]{slides/Methodlogy/10.pdf}}
	\only<11>{\includegraphics[width=\textwidth]{slides/Methodlogy/11.pdf}}
	\only<12>{\includegraphics[width=\textwidth]{slides/Methodlogy/12.pdf}}
	\only<13>{\includegraphics[width=\textwidth]{slides/Methodlogy/13.pdf}}
	\only<14>{\includegraphics[width=\textwidth]{slides/Methodlogy/14.pdf}}
	\only<15>{\includegraphics[width=\textwidth]{slides/Methodlogy/15.pdf}}
	\only<16>{\includegraphics[width=\textwidth]{slides/Methodlogy/16.pdf}}
	\only<17>{\includegraphics[width=\textwidth]{slides/Methodlogy/17.pdf}}
	\only<18>{\includegraphics[width=\textwidth]{slides/Methodlogy/18.pdf}}
	\only<19>{\includegraphics[width=\textwidth]{slides/Methodlogy/19.pdf}}
	\only<20>{\includegraphics[width=\textwidth]{slides/Methodlogy/20.pdf}}
	\only<21>{\includegraphics[width=\textwidth]{slides/Methodlogy/21.pdf}}
	\only<22>{\includegraphics[width=\textwidth]{slides/Methodlogy/22.pdf}}
	\only<23>{\includegraphics[width=\textwidth]{slides/Methodlogy/23.pdf}}
	\only<24>{\includegraphics[width=\textwidth]{slides/Methodlogy/24.pdf}}
	\only<25>{\includegraphics[width=\textwidth]{slides/Methodlogy/25.pdf}}
	\only<26>{\includegraphics[width=\textwidth]{slides/Methodlogy/26.pdf}}
	
	\only<27>{\includegraphics[width=\textwidth]{slides/Methodlogy/23.pdf}}
	\only<28>{\includegraphics[width=\textwidth]{slides/Methodlogy/24.pdf}}
	\only<29>{\includegraphics[width=\textwidth]{slides/Methodlogy/25.pdf}}
	\only<30>{\includegraphics[width=\textwidth]{slides/Methodlogy/26.pdf}}
	
	\only<31>{\includegraphics[width=\textwidth]{slides/Methodlogy/31.pdf}}
\end{frame}
	
	\section{Hybrid 8-bit Floating-Point and 4-bit Logarithmic Computation}
\tableofcontents[currentsection]

\begin{frame}{Spike-by-Spike Neural Network}
	\begin{columns}[c]  % Top alignment for the columns
		
		% Left Column for top left image
		\begin{column}{0.5\textwidth}
			\uncover<1->{  % Uncover the top left image first
				\begin{figure}
					\centering
					\includegraphics[width=\textwidth]{../chapters/sbs_accelerator/figures/SbS_layer.pdf}
					\caption{SbS inference}
				\end{figure}
			}
		\end{column}
		
		% Right Column for top right image
		\begin{column}{0.5\textwidth}
			\uncover<2->{  % Uncover the top right image second
				\begin{figure}
					\centering
					\includegraphics[width=0.9\textwidth]{../chapters/sbs_accelerator/figures/sbs_robustnes.pdf}
					\caption{Robustness of SbS NN versus CNN}
				\end{figure}
			}
		\end{column}
		
	\end{columns}
	\vspace{10mm}
	% Bottom full-width equation
	\uncover<3->{  % Uncover the equation last across the full width of the slide
		\[
		h_\mu^{new}(i) = \frac{1}{1+\epsilon} \left(h_\mu(i) + \epsilon \frac{h_\mu(i) W(s_t|i) }{\sum_j h_\mu(j) W(s_t|j)} \right) 
		\]
	}
\end{frame}

\begin{frame}{HW/SW Co-Design and Deployment Framework}

			\begin{figure}
				\includegraphics[width=0.75\textwidth]{slides/figures/sbs_software_component.pdf}
				\caption{System-level overview of the embedded software architecture}
			\end{figure}

\end{frame}

\begin{frame}{HW/SW Co-Design and Deployment Framework}

			\begin{figure}
				\includegraphics[width=0.75\textwidth]{./slides/figures/sbs_hw.pdf} % Adjust the filename
				\caption{System-level hardware architecture with scalable number of heterogeneous processing units}
			\end{figure}

\end{frame}

\begin{frame}{Hybrid Dot-Product Approximation}
	\begin{columns}[t] % The [T] option aligns the tops of the columns
		% Left column for equations
		\begin{column}{0.5\textwidth}

			\begin{equation}
			r_{\mu}\left(s_t\right)=\sum_{j=0}^{N-1}h_{\mu}(j)W(s_t|j)
			\end{equation}
			\vspace{4mm} 
			\begin{equation}
			E_{\min}=\log _2(\min_{\forall i}(W(i)))
			\end{equation}
			\vspace{4mm} 
			\begin{equation}
			N_E=\lceil\log_2(|E_{\min}|)\rceil
			\end{equation}
			\vspace{4mm} 
			\begin{equation}
			N_W=N_E + N_M
			\end{equation}
		\end{column}
		
		% Right column for the image
		\begin{column}{0.5\textwidth}
			\begin{figure}
				\centering
				\includegraphics[width=0.9\textwidth]{../chapters/sbs_accelerator/figures/dot-product_unit.pdf} % Adjust the filename
				\caption{Dot-product hardware module}
			\end{figure}
		\end{column}
	\end{columns}
\end{frame}

\begin{frame}{Dot-Product with Standard Floating-Point (IEEE 754)}
	\begin{figure}
		\centering
		\includegraphics[width=0.4\columnwidth]{../chapters/sbs_accelerator/figures/dot_product_float.pdf}
		\caption{Dot-product hardware module with standard floating-point computation}
	\end{figure}
	
	\vfill % Add vertical space to push the equation to the bottom
	
	% Equation at the bottom
	\[
	 L_{f32}=10N+9
	\]
\end{frame}

\begin{frame}{Dot-Product with Hybrid Custom Floating-Point Approximation}
	\begin{figure}
		\centering
		\includegraphics[width=0.4\columnwidth]{../chapters/sbs_accelerator/figures/dot_product.pdf}
		\caption{Dot-product hardware module with hybrid custom floating-point approximation}
	\end{figure}
	
	\vfill % Add vertical space to push the equation to the bottom
	
	% Equation at the bottom
	\[
	L_{custom}=2N+11
	\]
\end{frame}

\begin{frame}{Dot-product with Hybrid Logarithmic Approximation}
	\begin{figure}
		\centering
		\includegraphics[width=0.4\columnwidth]{../chapters/sbs_accelerator/figures/dot_product_log.pdf}
		\caption{Dot-product hardware module with hybrid logarithmic approximation}
	\end{figure}
	
	\vfill % Add vertical space to push the equation to the bottom
	
	% Equation at the bottom
	\[
	L_{custom}=2N+7
	\]
\end{frame}

\begin{frame}{Deployment with Standard Floating-Point}
	\begin{columns}
		% Left Column
		\begin{column}{0.5\textwidth}
			% Top left image
			\begin{minipage}[c][.45\textheight][c]{\linewidth}
				\centering
				\begin{figure}
				\includegraphics[width=0.75\linewidth]{./slides/figures/sbs_hw_experimental.pdf} % Adjust path and size as needed
				\caption{System overview of the top-level architecture with 8 processing units}
				\end{figure}
				\pause
			\end{minipage}
			
			% Bottom left image
			\begin{minipage}[c][.45\textheight][c]{\linewidth}
				\centering
				\begin{figure}
				\includegraphics[width=0.75\linewidth]{../chapters/sbs_accelerator/figures/latency_sw.pdf} % Adjust path and size as needed
				\caption{Computation on embedded CPU}
				\end{figure}
				\pause
			\end{minipage}
		\end{column}
		
		% Right Column
		\begin{column}{0.5\textwidth}
			% Top right image
			\begin{minipage}[c][.45\textheight][c]{\linewidth}
				\centering
				\begin{figure}
				\includegraphics[width=0.75\linewidth]{../chapters/sbs_accelerator/figures/latency_pu_fp.pdf} % Adjust path and size as needed
				\caption{Performance of processing units with standard floating-point with \textbf{acceleration of 10.7X} }
				\end{figure}
				\pause
			\end{minipage}
			
			% Bottom right image
			\begin{minipage}[c][.45\textheight][c]{\linewidth}
				\centering
				\begin{figure}
				\includegraphics[width=0.75\linewidth]{../chapters/sbs_accelerator/figures/latency_fp_cycle.pdf} % Adjust path and size as needed
				\caption{Performance bottleneck of cyclic computation on processing units with standard floating-point}
				\end{figure}
			\end{minipage}
		\end{column}
	\end{columns}
\end{frame}

\begin{frame}{Deployment with Custom Floating-Point}
	\begin{columns}[c] % The [T] option aligns the tops of the columns
		
		% Left column for the first image
		\begin{column}<1->{0.5\textwidth}
			\begin{figure}
				\includegraphics[width=0.75\textwidth]{../chapters/sbs_accelerator/figures/latency_cfp_cycle.pdf}
				 % Adjust the filename
				\caption{Performance on processing units with hybrid \textbf{8-bit floating-point} with \textbf{acceleration of 20.5X}}
			\end{figure}
		\end{column}
		
		% Right column for the second image
		\begin{column}<2->{0.5\textwidth}
			\begin{figure}
				\includegraphics[width=0.75\textwidth]{../chapters/sbs_accelerator/figures/latency_log_cycle.pdf}
				\caption{Performance of processing units with hybrid \textbf{4-bit logarithmic} with \textbf{acceleration of 20.5X}}
			\end{figure}
		\end{column}
		
	\end{columns}
\end{frame}

\begin{frame}{Quantization Impact: Noise Tolerance}
	\begin{columns}
		% First column
		\begin{column}{0.33\textwidth}
			\centering
			\begin{figure}
			\includegraphics[width=0.75\linewidth]{../chapters/sbs_accelerator/figures/accuracy_vs_noise_pu_fp.pdf} % Adjust path and size as needed
			\caption{ Noise tolerance with \textbf{32-bit floating-point}}
			\end{figure}
			\pause
		\end{column}
		
		% Second column
		\begin{column}{0.33\textwidth}
			\centering
			\begin{figure}
			\includegraphics[width=0.75\linewidth]{../chapters/sbs_accelerator/figures/accuracy_vs_noise_pu_cfp(4-bit-exponent_1-bit-mantissa).pdf} % Adjust path and size as needed
			\caption{ Noise tolerance with hybrid \textbf{8-bit floating-point}}
			\end{figure}
			\pause
		\end{column}
		
		% Third column
		\begin{column}{0.33\textwidth}
			\centering
			\begin{figure}
			\includegraphics[width=0.75\linewidth]{../chapters/sbs_accelerator/figures/accuracy_vs_noise_pu_log.pdf} % Adjust path and size as needed
			\caption{ Noise tolerance with hybrid \textbf{4-bit logarithmic}}
			\end{figure}
		\end{column}
	\end{columns}
\end{frame}

\begin{comment}

\begin{frame}[shrink=30]{Accelerator Implementations} % Title of the slide
	\begin{center} % Ensure the table is centered in the slide
		\begin{threeparttable}
			\caption{Accelerator implementations.} % Caption of the table
			\scriptsize % Reduce the font size of the table
			\begin{tabular}{lrrrrr}
				\toprule
				\textbf{Platform implementation} & \textbf{Power (W)} & \textbf{Clk (MHz)} & \textbf{Latency (ms)} & \textbf{Acceleration} & \textbf{Accuracy (\%)} \\
				\midrule
				Standard floating-point & 2.420 & 200 & 3.18 & 10.7x & 98.98 \\
				Hybrid floating-point 8-bit & 2.369 & 200 & 1.67 & 20.5x & 98.97 \\
				Hybrid Logarithmic 4-bit & 2.324 & 200 & 1.67 & 20.5x & 98.84 \\
				\bottomrule
			\end{tabular}
		\end{threeparttable}
	\end{center}
\end{frame}

\end{comment}
	
	\section{Hybrid 6-bit Floating-Point Computation}
\tableofcontents[currentsection]


\begin{frame}{Case Study: Applying Sensor Analytics to Structural Health Monitoring}
	\begin{columns}[t] % The [T] option aligns the tops of the columns
		
		% Left column for the first image
		\begin{column}<1->{0.5\textwidth}\centering
			\begin{figure}
				\includegraphics[width=0.5\textwidth]{../chapters/cnn_accelerator/figures/histograms/data_set.pdf} % Adjust the filename
				\caption{ Structural health monitoring, all lengths are in meters (m)}
			\end{figure}
		\end{column}
		
		% Right column for the second image
		\begin{column}<2->{0.5\textwidth}
			\begin{figure}
				\includegraphics[width=0.75\textwidth]{../chapters/cnn_accelerator/figures/models.pdf} % Adjust the filename
				\caption{ CNN-regression model for sensor analytics}
			\end{figure}
		\end{column}
		
	\end{columns}
\end{frame}


\begin{frame}{Evaluating Floating-Point Configurations: Speed, Power, and Hardware Utilization}
	% Using columns to arrange images vertically in two columns
	\begin{columns}[T] % Align columns at the top
		\begin{column}{0.5\textwidth}
			\centering
			\includegraphics[width=0.5\linewidth]{slides/figures/runtime_a.pdf} % Left top image
			\pause % Wait to reveal the next image
			
			\includegraphics[width=0.5\linewidth]{slides/figures/runtime_b.pdf} % Left middle image
			\pause % Wait to reveal the next image
			
			\includegraphics[width=0.5\linewidth]{slides/figures/runtime_c.pdf} % Left bottom image
			\pause % Wait to reveal the next image
		\end{column}
		
		\begin{column}{0.5\textwidth}
			\centering
			\includegraphics[width=0.5\linewidth]{slides/figures/acceleration_vs_cpu.pdf} % Right top image
			\pause % Wait to reveal the next image
			
			\includegraphics[width=0.5\linewidth]{slides/figures/power_reduction_vs_cpu.pdf} % Right middle image
			\pause % Wait to reveal the next image
			
			\includegraphics[width=0.8\linewidth]{slides/figures/resource_utilization.pdf} % Right bottom image
		\end{column}
	\end{columns}
\end{frame}


\begin{frame}{Quantization Impact: Error Histograms in Position Prediction}
	% Using columns to arrange images
	\begin{columns}[T] % Align columns at the top
		\begin{column}{0.5\textwidth}
			\centering
			\includegraphics[width=0.95\linewidth]{slides/figures/model_evaluation_a.pdf} % Left top image
			\pause % Wait to reveal the next image
			\includegraphics[width=0.95\linewidth]{slides/figures/model_evaluation_b.pdf} % Left middle image
			\pause % Wait to reveal the next image
			\includegraphics[width=0.95\linewidth]{slides/figures/model_evaluation_c.pdf} % Left bottom image
			\pause % Wait to reveal the next image
		\end{column}
		
		\begin{column}{0.5\textwidth}
			\centering
			\includegraphics[width=0.95\linewidth]{slides/figures/model_evaluation_d.pdf} % Right top image
			\pause % Wait to reveal the next image
			\includegraphics[width=0.95\linewidth]{slides/figures/model_evaluation_e.pdf} % Right middle image
			\pause % Wait to reveal the next image
			\includegraphics[width=0.95\linewidth]{slides/figures/model_evaluation_f.pdf} % Right bottom image
		\end{column}
	\end{columns}
\end{frame}

	
	\section{Conclusions}
\tableofcontents[currentsection]
\begin{frame}[t]
	\frametitle{Conclusion and Future Research}
	
	\begin{itemize} % The <+-> will make the items appear one-by-one
		\item \textbf{HW/SW Design Methodology for Low-Power Neural Network Acceleration with Hybrid Custom Floating-Point Arithmetic}
		\begin{itemize}
			\item<2-> Mimic standard floating-point acceleration in extreme edge devices
			\item<3-> Smallest CNN floating-point accelerator in the literature
			\item<4-> On-device iterative optimization with non-negativity constraints using 4-bit weights
		\end{itemize}
		\item \textbf{Technological Transformation}
		\begin{itemize}
			\item<5-> Inference was the beginning in TinyML. Low-precision floating-point is opening the door for a new generation of on-device training accelerators that are essential in the next phase of AI evolution.
		\end{itemize}
	\end{itemize}
\end{frame}
	
	\input{slides/publications.tex}
	
	\section*{}
\begin{frame}{}
	\centering
	\vfill
	\Large Thank You for Your Attention
	\vfill
\end{frame}
	
	% Start backup slides
\section*{Backup Slides}
\backupbegin
\begin{frame}{Spike-by-Spike Neural Network}

			\begin{figure}
				\includegraphics[width=0.9\textwidth]{../chapters/sbs_accelerator/figures/sbs_network.pdf} % Adjust the filename
				\caption{Spike-by-Spike (SbS) neural network architecture for handwritten digit classification task}
			\end{figure}

\end{frame}

\begin{frame}{SbS Processing Unit}
	\begin{columns}[c] % The [T] option aligns the tops of the columns
		
		% Left column for the first image
		\begin{column}<1->{0.5\textwidth}
			\begin{figure}
				\includegraphics[width=0.7\textwidth]{../chapters/sbs_accelerator/figures/sbs_conv.pdf} % Adjust the filename
				\caption{Convolution processing unit}
			\end{figure}
		\end{column}
		
		% Right column for the second image
		\begin{column}<2->{0.5\textwidth}
			\begin{figure}
				\includegraphics[width=0.9\textwidth]{../chapters/sbs_accelerator/figures/dot-product_unit.pdf} % Adjust the filename
				\caption{Dot-product hardware module}
			\end{figure}
		\end{column}
		
	\end{columns}
\end{frame}

\begin{frame}{Conv2D Tensor Processor}
	\begin{columns}[c] % The [T] option aligns the tops of the columns
		
		% Left column for the first image
		\begin{column}<1->{0.5\textwidth}
			\begin{figure}
				\includegraphics[width=0.7\textwidth]{../chapters/cnn_accelerator/figures/accelerator.pdf} % Adjust the filename
				\caption{High level architecture of tensor processor}
			\end{figure}
		\end{column}
		
		% Right column for the second image
		\begin{column}<2->{0.5\textwidth}
			\begin{figure}
				\includegraphics[width=0.9\textwidth]{../chapters/cnn_accelerator/figures/accelerator_buffers.pdf} % Adjust the filename
				\caption{ On-chip memory buffers}
			\end{figure}
		\end{column}
		
	\end{columns}
\end{frame}

\begin{frame}{Tensor Processor Setup Data Frame}
	
	\begin{figure}
		\includegraphics[width=\textwidth]{../figures/setup_transaction_buffer_stream.pdf}
		\caption{Setup transaction buffer stream}
	\end{figure}
	
\end{frame}


\begin{frame}{Embedded System Architecture with Tensor Processor}
	\begin{figure}
		\includegraphics[width=0.75\textwidth]{../chapters/cnn_accelerator/figures/system_design.pdf} % Adjust the filename
		\caption{Embedded system architecture}
	\end{figure}
\end{frame}

\begin{frame}
	\frametitle{Comparison with Related Work with Tensor Processor} % optional, remove or leave empty if no title is desired
	\begin{center}
		\includegraphics[width=\textwidth]{slides/figures/cnn_related_work.pdf} % Adjust the width as needed
	\end{center}
\end{frame}

\backupend
	
\end{document}
