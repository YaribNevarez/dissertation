\section{Hybrid 6-bit Floating-Point Computation}
\tableofcontents[currentsection]

\begin{frame}{Convolution Operation}
	\begin{figure}
		\centering
		\includegraphics[width=0.5\textwidth]{../figures/convolution.pdf}
		\caption{ Two dimensional convolution operation}
	\end{figure}
	
	%\pause % Pause for the equation to appear after the image
	
	% Equation at the bottom
	{\scriptsize
		\[
Conv2D\left(W,b,h\right)_{i,j,o}=\sum_{k,l,m}^{K,L,M} h_{(i+k,j+l,m)} W_{(o,k,l,m)}+b_{o}
		\]
	}
\end{frame}

\begin{frame}{HW/SW Co-Design and Deployment Framework}
	\begin{columns}[c] % The [T] option aligns the tops of the columns
		
		% Left column for the first image
		\begin{column}{0.5\textwidth}
			\begin{figure}
				\includegraphics[width=\textwidth]{../chapters/cnn_accelerator/figures/sw_stack.pdf}
				\caption{ High level embedded software architecture}
			\end{figure}
		\end{column}
		
		% Right column for the second image
		\begin{column}{0.5\textwidth}
			\begin{figure}
				\includegraphics[width=0.8\textwidth]{../chapters/cnn_accelerator/figures/system_design.pdf} % Adjust the filename
				\caption{ Base embedded system architecture}
			\end{figure}
		\end{column}
		
	\end{columns}
\end{frame}

\begin{frame}{Tensor Processor}
	\begin{columns}[c] % The [T] option aligns the tops of the columns
		
		% Left column for the first image
		\begin{column}<1->{0.5\textwidth}
			\begin{figure}
				\includegraphics[width=0.7\textwidth]{../chapters/cnn_accelerator/figures/accelerator.pdf} % Adjust the filename
				\caption{High level architecture of tensor processor}
			\end{figure}
		\end{column}
		
		% Right column for the second image
		\begin{column}<2->{0.5\textwidth}
			\begin{figure}
				\includegraphics[width=0.9\textwidth]{../chapters/cnn_accelerator/figures/accelerator_buffers.pdf} % Adjust the filename
				\caption{ On-chip memory buffers}
			\end{figure}
		\end{column}
		
	\end{columns}
\end{frame}

\begin{frame}{On-Chip Memory}
	\begin{columns}[c] % The [T] option aligns the tops of the columns
		% Left column for equations
		\begin{column}{0.5\textwidth}

			\vspace{1mm}
			\begin{equation}
			TP_{M}=TP_B+V_{M}
			\end{equation}
			\vspace{1mm} 
			\begin{equation}
			TP_{B}=Input_{M}+Filter_{M}+Bias_{M}
			\end{equation}
			\vspace{1mm} 
			\begin{equation}
			Input_{M}=K_{H}W_{I}C_{I}BitSize_{I}
			\end{equation}
			\vspace{1mm} 
			\begin{equation}
			Filter_{M}=C_{I}K_{W}K_{H}C_{O}BitSize_{F}
			\end{equation}
			\vspace{1mm} 
			\begin{equation}
			Bias_{M}=C_{O}BitSize_{B}
			\end{equation}
			\vspace{1mm} 
			\begin{equation}
			C_{O}=\frac{TP_{M}-V_{M}-K_{H}W_{I}C_{I}BitSize_{I}}{C_{I}K_{W}K_{H}BitSize_{F}+BitSize_{B}}
			\end{equation}
		\end{column}
		
		% Right column for the image
		\begin{column}{0.5\textwidth}
			\begin{figure}
				\centering
				\includegraphics[width=0.9\textwidth]{../chapters/cnn_accelerator/figures/accelerator_buffers.pdf} % Adjust the filename
				\caption{ On-chip memory buffers}
			\end{figure}
		\end{column}
	\end{columns}
\end{frame}


\begin{frame}{Dot-Product with Hybrid Floating-Point 6-bit}
	\begin{columns}[c] % The [T] option aligns the tops of the columns
		
		% Left column for the first image
		\begin{column}<1->{0.5\textwidth}\centering
			\begin{figure}
				\includegraphics[width=0.75\textwidth]{../chapters/cnn_accelerator/figures/dot-product_unit.pdf} % Adjust the filename
				\caption{\scriptsize Dot-product}
			\end{figure}
		\end{column}
		
		% Right column for the second image
		\begin{column}<2->{0.5\textwidth}
			\begin{figure}
				\includegraphics[width=0.5\textwidth]{../chapters/cnn_accelerator/figures/multiplier.pdf} % Adjust the filename
				\caption{Multiply-accumulate design}
			\end{figure}
		\scriptsize
		\[ L_{hf}=N+7 \]
		\end{column}
		
	\end{columns}
\end{frame}

\begin{frame}{Custom Floating-Point Quantization}
	\begin{figure}
		\centering
		\includegraphics[width=0.6\columnwidth]{slides/algorithm_fp_2.pdf}
	\end{figure}
		
\end{frame}


\begin{frame}{Case Study: Applying Sensor Analytics to Structural Health Monitoring}
	\begin{columns}[c] % The [T] option aligns the tops of the columns
		
		% Left column for the first image
		\begin{column}<1->{0.5\textwidth}\centering
			\begin{figure}
				\includegraphics[width=0.5\textwidth]{../chapters/cnn_accelerator/figures/histograms/data_set.pdf} % Adjust the filename
				\caption{ Structural health monitoring, all lengths are in meters (m)}
			\end{figure}
		\end{column}
		
		% Right column for the second image
		\begin{column}<2->{0.5\textwidth}
			\begin{figure}
				\includegraphics[width=0.75\textwidth]{../chapters/cnn_accelerator/figures/models.pdf} % Adjust the filename
				\caption{ CNN-regression model for sensor analytics}
			\end{figure}
		\end{column}
		
	\end{columns}
\end{frame}

\begin{frame}{Training}
	% Sequential appearance of images 2x2
	\begin{columns}
		\column{0.5\textwidth}
		\centering
		\includegraphics[width=0.65\linewidth]{slides/algorithm_early_stop_1.pdf} % Top left image
		\pause % Wait to reveal the next image
		
		\column{0.5\textwidth}
		\centering
		\includegraphics[width=0.65\linewidth]{slides/figures/training_iterative_early_stop.pdf} % Top right image
		\pause % Wait to reveal the next image
	\end{columns}
	
	\begin{columns}
		\column{0.5\textwidth}
		\centering
		\includegraphics[width=0.65\linewidth]{slides/OnMiniBatchUpdate_2.pdf} % Bottom left image
		\pause % Wait to reveal the next image
		
		\column{0.5\textwidth}
		\centering
		\includegraphics[width=0.65\linewidth]{slides/figures/QAT.pdf} % Bottom right image
	\end{columns}
\end{frame}

\begin{frame}{Quantization Impact: Error Histograms in Position Prediction}
	% Using columns to arrange images
	\begin{columns}[T] % Align columns at the top
		\begin{column}{0.5\textwidth}
			\centering
			\includegraphics[width=0.95\linewidth]{slides/figures/model_evaluation_a.pdf} % Left top image
			\pause % Wait to reveal the next image
			\includegraphics[width=0.95\linewidth]{slides/figures/model_evaluation_b.pdf} % Left middle image
			\pause % Wait to reveal the next image
			\includegraphics[width=0.95\linewidth]{slides/figures/model_evaluation_c.pdf} % Left bottom image
			\pause % Wait to reveal the next image
		\end{column}
		
		\begin{column}{0.5\textwidth}
			\centering
			\includegraphics[width=0.95\linewidth]{slides/figures/model_evaluation_d.pdf} % Right top image
			\pause % Wait to reveal the next image
			\includegraphics[width=0.95\linewidth]{slides/figures/model_evaluation_e.pdf} % Right middle image
			\pause % Wait to reveal the next image
			\includegraphics[width=0.95\linewidth]{slides/figures/model_evaluation_f.pdf} % Right bottom image
		\end{column}
	\end{columns}
\end{frame}

\begin{frame}{Evaluating Floating-Point Configurations: Speed, Power, and Hardware Utilization}
	% Using columns to arrange images vertically in two columns
	\begin{columns}[T] % Align columns at the top
		\begin{column}{0.5\textwidth}
			\centering
			\includegraphics[width=0.5\linewidth]{slides/figures/runtime_a.pdf} % Left top image
			\pause % Wait to reveal the next image
			
			\includegraphics[width=0.5\linewidth]{slides/figures/runtime_b.pdf} % Left middle image
			\pause % Wait to reveal the next image
			
			\includegraphics[width=0.5\linewidth]{slides/figures/runtime_c.pdf} % Left bottom image
			\pause % Wait to reveal the next image
		\end{column}
		
		\begin{column}{0.5\textwidth}
			\centering
			\includegraphics[width=0.5\linewidth]{slides/figures/acceleration_vs_cpu.pdf} % Right top image
			\pause % Wait to reveal the next image
			
			\includegraphics[width=0.5\linewidth]{slides/figures/power_reduction_vs_cpu.pdf} % Right middle image
			\pause % Wait to reveal the next image
			
			\includegraphics[width=0.8\linewidth]{slides/figures/resource_utilization.pdf} % Right bottom image
		\end{column}
	\end{columns}
\end{frame}

\begin{frame}
	\frametitle{Comparison with Related Work} % optional, remove or leave empty if no title is desired
	\begin{center}
		\includegraphics[width=\textwidth]{slides/figures/cnn_related_work.pdf} % Adjust the width as needed
	\end{center}
\end{frame}