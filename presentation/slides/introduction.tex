\section*{Introduction}
\begin{frame}{Introduction}
	\begin{itemize}
		\item<1-> \textbf{Transformation:} AI is undergoing a continual transformation, becoming increasingly ubiquitous as it integrates into low-power, extreme-edge devices.
		\item<2-> \textbf{Requirements:} Efficiency is imperative to ensure long-term sustainability and broad scalability.
		\item<3-> \textbf{Limitations:} Existing state-of-the-art methods, including fixed precision and aggressive quantization, have been explored but fail to adequately meet the critical requirements for quality preservation, versatility, and compatibility in AI applications.
		\item<4-> \textbf{Innovation:} Implementing sub-8-bit custom floating-point formats in neural computation addresses these challenges.
	\item<5-> \textbf{Future-proof:} Custom floating-point computation opens the door for the next phase of AI evolution, facilitating on-device training in extreme-edge devices.
	\end{itemize}
\end{frame}
	
	\section*{Goal and Objectives}
	\begin{frame}
		\frametitle{Goal and Objectives}
		\begin{itemize}
			\item<1-> \textbf{Goal}:
			To establish a sustainable, efficient, and universally applicable neural network acceleration technique that supports inference and facilitates on-device training in extreme edge devices, ensuring future-proof and broad applicability. This approach integrates hybrid custom floating-point (FP) computation.
			
			\vspace{10mm} 
			
			\item<2-> \textbf{Objectives}:
			\begin{itemize}
				\item<3-> Investigate custom FP computation
				\item<4-> Conduct performance evaluation
				\item<5-> Analyze impact
				\item<6-> Ensure cross-platform compatibility
			\end{itemize}
		\end{itemize}
	\end{frame}

	\begin{frame}
	\frametitle{Outline}
	\tableofcontents % This command automatically creates a table of contents based on the sections and subsections
	\end{frame}