\chapter{Conclusion and Outlook}
\label{chap.conclusion}
\minitoc
The use of \gls{ai} is entering a new era based on the use of ubiquitous embedded connected devices. The sustainability of this transformation requires the adoption of design techniques that reconcile accurate results with cost-effective system architectures. As such, improving the efficiency of \gls{ai} hardware engines as well as \gls{ml} portability must be considered.

In the emerging era of Industry 4.0, \gls{ml} algorithms yield the power of \gls{ai} to massively ubiquitous \gls{iot} devices. Applications in this field become smarter and more profitable as the availability of big data gets expanded, driving evolution of many aspects in science, industry, and daily life. However, state-of-the-art \gls{ml} algorithms, specially \gls{snn} and \gls{cnn}, represent elevated computational and energy costs. Therefore, hardware efficiency is one of the major goals to innovate compute engines as they are the machinery of the future.

Energy, performance, and chip-area are the key design concerns in computer systems. Considering the intrinsic error resilience of \gls{ml} algorithms, paradigms such as approximate computing come to the rescue by offering promising efficiency gains to assist the aforementioned design concerns. Approximation techniques are widely used in \gls{ml} algorithms at the model-structure as well as at the hardware processing level. However, state-of-the-art methods do not sufficiently address accelerator designs for \gls{ann}, in particular with \gls{fp} computation.

To sustain the continuous expansion of \gls{ml} applications on cost-effective compute devices, approximate computing will gradually transform from a design alternative to an essential prerequisite. This dissertation focuses on the investigation of design methodologies to exploit the intrinsic error resilience of \gls{ml} algorithms to optimize \gls{fp} inference in low-power embedded systems.

In this work, we accelerate SbS neural networks with a dot-product functional unit based on approximate computing that combinesthe advantages of custom floating-point and logarithmic representations. This approach reduces computational latency, memory footprint, and power dissipation while preserving classification accuracy. For output quality monitoring, we applied noise tolerance plots as an intuitive visual measure to provide insights into the accuracy degradation of SbS networks under different approximate processing effects. This plot revels inherent error resilience, hence, the possibilities for approximate processing.


The proposed approach is demonstrated with a design exploration flow on a Xilinx Zynq-7020 with a deployment of \gls{sbs} network for MNIST classification task. This implementation achieves up to $20.5\times$ latency enhancement, $8\times$ weight memory footprint reduction, and $12.35\%$ of energy efficiency improvement over the standard floating-point hardware implementation, this deployment incurs in less than $0.5\%$ of accuracy degradation. Furthermore, with noise amplitude of $50\%$ added on the input images, the \gls{sbs} network presents an accuracy degradation of less than $5\%$. To monitor the inference quality, the resulting noise tolerance plots demonstrate a sufficient \gls{qor} for minimal impact on the overall accuracy of the neural network under the effects of this approximation technique. These results suggest available room for further or more aggressive approximate processing approaches.


In summary, based on the relaxed need for fully accurate or deterministic computation of neural networks, approximate computing techniques allow substantial enhancement in processing efficiency with moderated accuracy degradation.

This chapter presents the Hybrid-Float6 quantization and its dedicated hardware accelerator for floating-point \gls{cnn} computation. Feature maps and weights are represented by 32-bit and 6-bit \gls{fp}, respectively. The 6-bit \gls{fp} format is composed of 1-bit sign, 4-bit exponent, and 1-bit mantissa. The 1-bit mantissa enables low-power \gls{mac} implementations by reducing the mantissa multiplication to a multiplexer-adder operation. The intrinsic error tolerance of neural networks is exploited to further reduce the hardware design with approximation. This approach improves latency, hardware area, and energy consumption. To preserve accuracy, a \gls{qat} training method is presented that, based on regularization effects can improve accuracy. A lightweight \gls{tp} implementing a pipelined vector dot-product is presented. For \gls{ml} compatibility/portability, the 6-bit \gls{fp} is wrapped in the standard floating-point format, which is automatically extracted by the proposed hardware. The hardware/software architecture is compatible with TensorFlow Lite. To evaluate the applicability of this approach, it is presented a \gls{cnn}-regression model for anomaly localization in a \gls{shm} application based on acoustic emissions. The embedded hardware/software framework is demonstrated on XC7Z007S as the smallest Zynq-7000 \gls{soc}, suitable for low-power \gls{iot} applications. The proposed architecture achieves a peak power efficiency and acceleration on convolution layers of \unit[5.7]{GFLOPS/s/W} and $48.3\times$, respectively.


This chapter presents the Hybrid-Float6 quantization and its dedicated hardware accelerator for floating-point \gls{cnn} computation. Feature maps and weights are represented by 32-bit and 6-bit \gls{fp}, respectively. The 6-bit \gls{fp} format is composed of 1-bit sign, 4-bit exponent, and 1-bit mantissa. The 1-bit mantissa enables low-power \gls{mac} implementations by reducing the mantissa multiplication to a multiplexer-adder operation. The intrinsic error tolerance of neural networks is exploited to further reduce the hardware design with approximation. This approach improves latency, hardware area, and energy consumption. To preserve accuracy, a \gls{qat} training method is presented that, based on regularization effects can improve accuracy. A lightweight \gls{tp} implementing a pipelined vector dot-product is presented. For \gls{ml} compatibility/portability, the 6-bit \gls{fp} is wrapped in the standard floating-point format, which is automatically extracted by the proposed hardware. The hardware/software architecture is compatible with TensorFlow Lite. To evaluate the applicability of this approach, it is presented a \gls{cnn}-regression model for anomaly localization in a \gls{shm} application based on acoustic emissions. The embedded hardware/software framework is demonstrated on XC7Z007S as the smallest Zynq-7000 \gls{soc}, suitable for low-power \gls{iot} applications. The proposed architecture achieves a peak power efficiency and acceleration on convolution layers of \unit[5.7]{GFLOPS/s/W} and $48.3\times$, respectively.

\section{Summary of Contributions}

In the field of \gls{snn}, this dissertation presents a hardware design methodology for low-power inference of \gls{sbs} neural networks targeting embedded applications. This \gls{ml} algorithm provides exceptional noise robustness and reduced complexity compared to conventional \gls{snn} with \gls{lif} mechanism. However, \gls{sbs} networks represent a memory footprint and a computational cost unsuitable for embedded applications. To address this problem, this work exploits the intrinsic error resilience of \gls{sbs} to improve performance and to reduce hardware complexity. More precisely, we design a vector dot-product module based on approximate computing with configurable quality using hybrid custom \gls{fp} and logarithmic number representations. This approach reduces computational run-time, memory footprint, and power dissipation while preserving inference accuracy. To demonstrate this approach, we address a design exploration flow with \gls{hls} on a \gls{fpga}. The proposed design reduces $20.5\times$ run-time and $8\times$ weight memory footprint, with less than $0.5\%$ of accuracy degradation without retraining on a handwritten digit classification task.

In the field of \gls{cnn}, this dissertation presents a hardware design methodology for low-power inference targeting sensor analytics applications. In this work, we present the \gls{hf6} quantization and its dedicated hardware processor. We propose an optimized \gls{fp} \gls{mac} hardware by reducing the mantissa multiplication to a multiplexer-adder operation. We exploit the intrinsic error tolerance of neural networks to further reduce the hardware design with approximation on the subnormal number computation. To preserve model accuracy, we present a \gls{qat} method, which in some cases improves accuracy. We demonstrate this concept in 2D convolution layers. We present a lightweight \gls{tp} implementing a pipelined vector dot-product. For \gls{ml} portability, the custom \gls{fp} representation is wrapped in the standard format, which is automatically extracted by the proposed hardware. The hardware/software architecture is integrated with \gls{tf} Lite. We evaluate the applicability of our approach with a \gls{cnn}-regression model for anomaly localization in a \gls{shm} application based on \gls{ae}. The embedded hardware/software framework is demonstrated on XC7Z007S as the smallest Zynq-7000 \gls{soc}. The proposed implementation achieves a peak power efficiency and acceleration of $5.7$ GFLOPS/s/W and $48.3\times$, respectively.

The outcome of this dissertation aims to contribute to the rise of a sustainable next generation of low-power \gls{fp} neural network processors with \gls{ml} portability as a design philosophy.

\section{Future Works}



