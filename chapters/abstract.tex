\chapter*{Abstract}
\thispagestyle{empty}

The expansion of \gls{ai} is addressing a new era characterized by omnipresent connected devices. To ensure the sustainability of this transformation, it is imperative to adopt design strategies that harmonize precise computational results with economically viable system architectures. Consequently, refining the efficiency and quality of \gls{ai} hardware engines stand as critical considerations in this evolving landscape. This necessitates a balanced approach that prioritizes energy-efficient computations, precise and reliable results, and seamless integration across various platforms and devices.

In the context of Industry 4.0, \gls{ml} algorithms are serving as the foundational enabler for the integration of \gls{ai} into \gls{iot} devices. These advancements are shaping applications to be more intelligent and economically rewarding. This transformation improves numerous domains, from scientific research to industrial processes and everyday living. However, this technological evolution comes with its own set of challenges. \gls{ml} algorithms pose significant computational and energy demands. Consequently, a central objective of this dissertation is to explore innovative methods for enhancing the hardware efficiency of computing engines.

Approximate computing techniques, such as quantization, exploit the inherent error resilience of \gls{ml} algorithms to address key design concerns in computer systems: energy efficiency, performance, and chip area. Quantization, which involves reducing the number of bits used to represent numbers, can significantly lower power consumption and data movement, thereby enhancing energy efficiency, and enables compact arithmetic units that save chip area. These techniques often yield computation acceleration due to reduced data sizes, which promotes faster, more parallel, and pipelined processing, particularly in specialized hardware accelerators designed for neural network computation. However, this approach introduces a trade-off between precision and model accuracy, necessitating proper hardware design methodologies. While state-of-the-art methods are advancing, significant research opportunities remain, especially for accelerators with custom \gls{fp} computation.

In this dissertation, a hardware design methodology is presented for low-power inference of \gls{sbs} neural networks for embedded applications, within the field of \glspl{snn}. Compared to conventional \glspl{snn} employing the \gls{lif} mechanism, \gls{sbs} neural networks are highlighted for their exceptional noise robustness and reduced complexity. However, despite their advantages, \gls{sbs} networks inherently possess a memory footprint and computational cost that makes them challenging for deployment in constrained devices. To solve this issue, this research leverage the intrinsic error resilience of \gls{sbs} models, aiming to enhance performance and reduce hardware complexity, while avoiding model quantization or retraining. Specifically, this research introduces a novel hardware \gls{mac} module designed to optimize the balance between computational accuracy and resource efficiency. This \gls{mac} module features configurable quality through a hybrid approach. It combines standard \gls{fp} number representations with a custom 8-bit \gls{fp} format, as well as a 4-bit logarithmic number representation. This design excludes the use of a sign bit, further contributing to the compact and efficient representation of numbers. This design enables the \gls{mac} module to be tailored to the specific resource constraints and performance requirements of a given application, thereby making \gls{sbs} neural networks possible for deployment in low-power, resource-constrained environments.

In the field of \glspl{cnn}, this dissertation presents a hardware design methodology for low-power inference, specifically targeting sensor analytics applications. Central to this work is the proposal of the \gls{hf6} quantization scheme and its dedicated hardware accelerator, designed to function as a Conv2D \gls{tp}. This quantization strategy employs a hybrid number representation, combining standard \gls{fp} and a 6-bit \gls{fp} format. This strategy allows for a highly optimized \gls{fp} \gls{mac}, reducing mantissa multiplication into a multiplexer-adder operation. This research introduces a \gls{qat} method that, in certain cases, offers beneficial regularization effects. The efficacy of this exploration is demonstrated with a regression model, which improves its accuracy despite the applied quantization. For \gls{ml} portability, the custom \gls{fp} representation is encapsulated within a standard format -- a design feature that the proposed hardware automatically processes. To validate interoperability of this approach, the hardware architecture is integrated with TensorFlow Lite, demonstrating compatibility with industry-standard \gls{ml} frameworks and affirming the potential for practical deployment in various sensing applications while maintaining compliance with established \gls{ml} infrastructure.


This dissertation addresses a pivotal challenge in the current technological landscape: the harmonization of computational accuracy with energy-efficient and compatible hardware solutions. This dissertation stands as a significant contribution towards the development of a sustainable next-generation of neural network processors, essential to empower the increasingly connected and intelligent world of tomorrow.