\usepackage[a-1b]{pdfx}
\usepackage{xmpincl}
\usepackage[inner=3.0cm,outer=2cm,top=2cm,bottom=2cm,includeheadfoot]{geometry}
\usepackage[onehalfspacing]{setspace}
%\setstretch{baselinestretch}
%\linespread{1.5} % line spacing 
 
\usepackage[headsepline,plainheadsepline]{scrpage2}
\let\endgraph\endgraf        % necessary for "chapterprefix" in \documentclass call

% Header and footer definition
\clearscrheadfoot            % clear header and footer
\ohead[\headmark]{\headmark} % chapter names at top
\automark[section]{chapter}
\pagestyle{scrheadings}      % choose page style
%
%% % try to solve the width problem in PDF/A
%\usepackage[T1]{fontenc}

\ofoot[\pagemark]{\pagemark} % page number on the upper right side
%--------- separate abstract and content with one blank page---------- %
\usepackage{afterpage}
\newcommand\blankpage{%
    \null
    \thispagestyle{empty}%
    \addtocounter{page}{-1}%
    \newpage}
%\usepackage[backend=biber]{biblatex}
% ------ Language and Text Options
\usepackage[english]{babel}
\usepackage[latin9]{inputenc}
%\usepackage{pdflatex}
%\usepackage{color}
\usepackage{float}
%\usepackage{xmpincl}
\usepackage{hyperref}

%******** complexity O()*******
\usepackage{flushend}
\renewcommand{\O}[1]{$\mathcal{O}(#1)$}


%*********inline list
\usepackage{enumerate}
\usepackage[inline,shortlabels]{enumitem}

%%********inkscape to latex*************
%\usepackage{pstricks}
%%\usepackage{pst-plot}

%\usepackage[pdftex,
%        pdftoolbar=true, hyperfootnotes=false,breaklinks=true,
%        pdfpagelabels=true,
%        bookmarks, bookmarksopen, bookmarksnumbered, bookmarksopenlevel=1,
%        pdfauthor={Yanqiu Huang},
%        pdfsubject={Dissertation},%
%        pdfkeywords={WSNs, Energy efficiency, Data compression},
%        pdftitle={Dissertation,Yanqiu Huang,Uni-Bremen,2017},
%        pdfstartview={FitH},
%        pdfborder={0 0 0},  %Farbe aller Linkumrandung auf Wei� gesetzt
%        plainpages=false]{hyperref}
%\def\UrlBreaks{\do\/\do-}

% this package is used to remove the double printed math symbole problem
\usepackage{newtxtext,newtxmath}

\usepackage{etoc}
% ------ Math Packages
\usepackage{amsfonts}
\usepackage{amsmath}
\usepackage{bm}
%\usepackage{amssymb}
\newcommand\numberthis{\addtocounter{equation}{1}\tag{\theequation}}
%\usepackage{mathtools}
\usepackage[detect-none]{siunitx}
\usepackage[capitalize]{cleveref}
% ------ Source Code
\usepackage{listings}
%\usepackage{enumerate}
% ----- Floating Opjects
\usepackage{graphicx}
\usepackage{subfig}
\usepackage{xcolor}

%***packages for inputing the tikz pictures**********
\newlength\figureheight 
\newlength\figurewidth

\usepackage{pgfplots}
%******** change the tick, label size of the tikz figures********
\pgfplotsset{every axis/.append style={
		label style={font=\small},
		tick label style={font=\small}  
	}}
	
\usepackage{units}	
\usepackage{color}
	
\usepackage{epstopdf}
\usepackage{rotating}
\usepackage{multirow}		% -- multirow and multicolumn
\usepackage{booktabs}		% -- professional tables in LaTeX ,
\usepackage{longtable}
\usepackage{capt-of}
%\usepackage[bf]{caption}		% -- customise Captions
\usepackage{caption}		% -- customise Captions
\usepackage{placeins} 		% -- FloatBarrier
\usepackage{xspace}			% -- for a better layout of spaces
\usepackage{tabularx}

% ------ Table of Contents
\usepackage{minitoc}
%\dominitoc % for minitoc, default left-bounded
\setcounter{tocdepth}{2}
\setcounter{secnumdepth}{2}

% Packages for acronyms and list of symbols
%\usepackage{acronym}
\usepackage{makeidx}
\usepackage[section,style=alttree, sort=use]{glossaries}
\glssetwidest{ADCADCA}% widest name
\renewcommand*{\glsnamefont}[1]{\textmd{#1}}
\GlsSetQuote{+}
\usepackage{ifthen}

% *** SPECIALIZED LIST PACKAGES ***
\usepackage[flushleft]{threeparttable}
\usepackage[ruled, vlined, linesnumbered]{algorithm2e}
%\usepackage{algpseudocode}
%\usepackage[ruled,vlined,linesnumbered]{algorithm2e}
\usepackage{algorithmic}
\newtheorem{theorem}{Theorem}
\newenvironment{proof}[1][Proof]{\begin{trivlist}
\item[\hskip \labelsep {\bfseries #1}]}{\end{trivlist}}
%------ package for list of symbols
%\usepackage{glossaries}
\usepackage[]{appendix}
\usepackage{dirtree}



% Define the colors for the listings package
\definecolor{codegreen}{rgb}{0,0.6,0}
\definecolor{codegray}{rgb}{0.5,0.5,0.5}
\definecolor{codepurple}{rgb}{0.58,0,0.82}
\definecolor{backcolour}{rgb}{0.95,0.95,0.92}

% Style for the C++ listings
\lstdefinestyle{mystyle}{
	backgroundcolor=\color{backcolour},   
	commentstyle=\color{codegreen},
	keywordstyle=\color{magenta},
	numberstyle=\tiny\color{codegray},
	stringstyle=\color{codepurple},
	basicstyle=\ttfamily\small,
	breakatwhitespace=false,         
	breaklines=true,                 
	captionpos=b,                    
	keepspaces=true,                 
	numbers=left,                    
	numbersep=5pt,                  
	showspaces=false,                
	showstringspaces=false,
	showtabs=false,                  
	tabsize=2
}

\lstset{style=mystyle}